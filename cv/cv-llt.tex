\documentclass[a4paper,skipsamekey,11pt,english]{curve}

% Default biblatex style used for the publication list is APA6. If you wish to use a different style or pass other options to biblatex you can change them here. 
\PassOptionsToPackage{style=ieee,sorting=ydnt,uniquename=init,defernumbers=true}{biblatex}

% Most commands and style definitions are in settings.sty.
\usepackage{settings}

\DefineBibliographyStrings{english}{url={\textsc{url}}}

%% Only needed if you want a Publication List
\addbibresource{own-bib.bib}

%% You can specify multiple names like this, especially if you have changed your name or if you need to highlight multiple authors. See items 6–9 in the example "Journal Articles" output.

\mynames{GIL/Jey\bibnamedelima PUGET}
%% MAKE SURE THERE IS NO SPACE AFTER THE FINAL NAME IN YOUR \mynames LIST

% Change the fonts if you want
\ifxetexorluatex % If you're using XeLaTeX or LuaLaTeX
  \usepackage{fontspec} 
  %% You can use \setmainfont etc; I'm just using these font packages here because they provide OpenType fonts for use by XeLaTeX/LuaLaTeX anyway
  \usepackage[p,osf,swashQ]{cochineal}
  \usepackage[medium,bold]{cabin}
  \usepackage[varqu,varl,scale=0.9]{zi4}
\else % If you're using pdfLaTeX or latex
  \usepackage[T1]{fontenc}
  \usepackage[p,osf,swashQ]{cochineal}
  \usepackage{cabin}
  \usepackage[varqu,varl,scale=0.9]{zi4}
\fi

% Change the page margins if you want
% \geometry{left=1cm,right=1cm,top=1.5cm,bottom=1.5cm}

% Change the colours if you want
% \definecolor{SwishLineColour}{HTML}{00FFFF}
% \definecolor{MarkerColour}{HTML}{0000CC}

% Change the item prefix marker if you want
% \prefixmarker{$\diamond$}

%% Photo is only shown if "fullonly" is included
\includecomment{fullonly}
% \excludecomment{fullonly}


%%%%%%%%%%%%%%%%%%%%%%%%%%%%%%%%%%%%%%


\leftheader{%
  {\LARGE\bfseries\sffamily Jey Puget Gil, Research and Teaching Associate}

  \makefield{\faTags}{\texttt{\#databases, \#knowledge, \#graphs, \#data, \#evolution, \#workflow}}

  \makefield{\faEnvelope[regular]}{\href{mailto:jey.puget-gil@liris.cnrs.fr}{\texttt{jey.puget-gil[at]liris.cnrs.fr}}}
  \makefield{\faHome}{\href{https://www.openstreetmap.org/relation/120965}{\texttt{Lyon, FRANCE}}}
  % fontawesome5 doesn't have the X icon so we use
  % the simpleicons package here instead; but some 
  % font size adjustment might be needed
  % \makefield{{\scriptsize\simpleicon{x}}}{\!\href{https://x.com/overleaf_example}{\texttt{@overleaf\_example}}}

  \makefield{\faOrcid}{\href{https://orcid.org/0009-0006-6198-7488}{\texttt{0009-0006-6198-7488}}}
  \makefield{\faResearchgate}{\href{https://www.researchgate.net/profile/Jey-Puget-Gil}{\texttt{Jey-Puget-Gil}}}

  % You can use a tabular here if you want to line up the fields.
  \makefield{\faGlobe}{\url{http://hemoreg.me/}}
  \makefield{\faGithub}{\href{https://github.com/JPugetGil}{\texttt{JPugetGil}}}  
}

\rightheader{~}
\begin{fullonly}
\photo[r]{photo}
\photoscale{0.13}
\end{fullonly}

\title{Curriculum Vitae - Jey PUGET GIL}

\begin{document}
\makeheaders[c]

\makerubric{employment}
\makerubric{education}

% If you're not a researcher nor an academic, you probably don't have any publications; delete this line.
%% Sometimes when a section can't be nicely modelled with the \entry[]... mechanism; hack our own and use \input NOT \makerubric
%% Sometimes when a section can't be nicely modelled with the \entry[]... mechanism; hack our own
\makerubrichead{Research Publications}

%% Assuming you've already given \addbibresource{own-bib.bib} in the main doc. Right? Right???
\nocite{*}

%% If you just want everything in one list
\printbibliography[heading={none}]

% \printbibliography[heading={subbibliography},title={Journal Articles},type=article]

% \printbibliography[heading={subbibliography},title={Conference Proceedings},type=inproceedings]

% \printbibliography[heading={subbibliography},title={Books and Chapters},filter={booksandchapters}]



\makerubrichead{Teaching}

\begin{tabularx}{\textwidth}{c c X c c}
    \textbf{Year} & \textbf{Degree} & \textbf{Course}   & \textbf{Tutorial}     & \textbf{Practical Work}   \\ \hline
    \multirow{5}{*}{2025-2026} & Bachelor's & Advanced Databases & 20h          & 18h         \\ \cline{2-5}
     & Bachelor's & Object Oriented Programming & 40.5h         & 19.5h         \\ \cline{2-5}
     & Bachelor's & Formal languages theory, Classical logic & 21h         & 12h         \\ \cline{2-5}
     & Bachelor's & Functional programming & 9h         & 12h         \\ \cline{2-5}
     & Bachelor's & Databases and Web & 18h         & 45h         \\ \hline
    %
    \multirow{3}{*}{2024-2025} & Bachelor's & Computer Architecture & 1.5h        & 0h          \\ \cline{2-5}
     & Bachelor's & Advanced Databases & 20h         & 26h         \\ \cline{2-5}
     & Bachelor's & Databases and Web & 18h         & 0h          \\ \hline
    % 
    \multirow{4}{*}{2023-2024} & Master's & Introduction to Research & 3.5h        & 0h          \\ \cline{2-5}
     & Bachelor's & Advanced Databases & 0h          & 18h         \\ \cline{2-5}
     & Bachelor's & Web Design and Programming & 0h          & 20.5h       \\ \cline{2-5}
     & Bachelor's & Human-Computer Interaction and Software Ergonomics & 0h          & 30h         \\ \hline
    % 
    \multirow{2}{*}{2022-2023} & Master's & Synchronous and Multi-device Web Technologies & 0h          & 18h         \\ \cline{2-5}
     & Bachelor's & Human-Computer Interaction and Software Ergonomics       & 0h          & 28.5h       \\ \hline
\end{tabularx}

\makerubric{skills}
\makerubric{misc}

% \makerubric{referee}
\makerubrichead{References}

\begin{tabularx}{\textwidth}{@{}X X@{}}
\textbf{Emmanuel COQUERY}\par
Associate Professor\par
Université Claude Bernard Lyon 1,\par 
43 Bd du 11 novembre 1918, 69622 Villeurbanne Cedex.\par 
\makefield{\faEnvelope}{\url{emmanuel.coquery@liris.cnrs.fr}}
% & 
% \textbf{Prof Y}\par
% Professor\par
% ABC University,\par 
% Address.\par 
% \makefield{\faEnvelope}{\url{abc@def.edu}}
\\
\end{tabularx}


\end{document}